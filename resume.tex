\documentclass[letterpaper,11pt]{article}
\usepackage{latexsym}
\usepackage[empty]{fullpage}
\usepackage{titlesec}
\usepackage[usenames,dvipsnames]{color}
\usepackage[inline]{enumitem}

% Enable hyperlinks
\usepackage[pdftex]{hyperref}

% Set margins
\usepackage[left=0.5in,top=0.5in,right=0.5in,bottom=0.5in]{geometry}

% Used to set font
\usepackage{kpfonts}

% Fix ATS
\input{glyphtounicode}
\pdfgentounicode=1

% Bullet formatting
\AtBeginDocument{
  \setlist[itemize]{
    itemsep=-2pt,
    leftmargin=*,
    before=\small\vspace{-\baselineskip}\vspace{5pt},
    after=\vspace{-\baselineskip}\vspace{5pt}
  }
}

\raggedbottom
\raggedright
\setlength{\tabcolsep}{0in}
\setlength{\footskip}{5pt}

% Sections formatting
\titleformat{\section}{
  \vspace{-9pt}\scshape\raggedright\Large
}{}{0em}{}[\color{black}\titlerule \vspace{-5pt}]

% Custom commands for subheadings
\newcommand{\subheading}[5]{
  \ifx&#2&%
  \textbf{#1} \hfill \textbf{#3} \\
  \textit{\small #4} \hfill \textit{\small #5} \\
  \else
  \textbf{#1}, \emph{#2} \hfill \textbf{#3} \\
  \textit{\small #4} \hfill \textit{\small #5} \\
  \fi
}

\newcommand{\subheadingLine}[3]{
  \ifx&#2&%
  \textbf{#1} \hfill \textbf{#3} \\
  \else
  \textbf{#1}, \emph{#2} \hfill \textbf{#3} \\
  \fi
}

\begin{document}

\setlist[itemize]{itemsep=-2pt, leftmargin=*, before=\small\vspace{-\baselineskip}\vspace{5pt},
    after=\vspace{-\baselineskip}\vspace{5pt}}

\begin{center}
    \Huge{\textbf{Conner Rose}} \\
    \small
    \begin{center}
        \href{https://linkedin.com/in/ConnerRose}{linkedin.com/in/ConnerRose}
        \labelitemi
        \ \href{https://github.com/ConnerRose}{github.com/ConnerRose}
        \labelitemi
        \ \href{mailto:conner.n.rose@gmail.com}{conner.n.rose@gmail.com}
        \labelitemi
        \ \href{tel:+15176481359}{(517) 648-1359}
    \end{center}
\end{center}

\section{Education}
\subheading{University of Michigan}{Ann Arbor, MI}{August 2022 -- May 2025}
{B.S.E. in Computer Science, Completing Requirements for B.S. in Honors Mathematics}{GPA: 3.88/4.0}
\begin{itemize}
    \item \textbf{CS Coursework}: Programming and Data Structures, Data Science for
          Engineers, Data Structures and Algorithms, Discrete Mathematics, Machine
          Learning, Algorithm and Computation Theory, Computer Organization, Web Systems
    \item \textbf{Mathematics Coursework}: Calculus I-IV, Linear Algebra, Combinatorics
          and Graph Theory, Probability, Real Analysis, Probability Theory, Theoretical
          Statistics
    \item \textbf{Extracurricular}: Traders at Michigan, Quantitative Investment Society,
          Michigan Hackers, Mathematics Club, MRun
\end{itemize}

\section{Experience}
\subheading{Bloomberg L.P.}{New York, NY}{May -- August 2023}
{CTO Office Intern - Compute Architecture and OSPO}{}
\begin{itemize}
    \item Designed automated access revocation system using \textbf{Python} and
          \textbf{LDAP}, deployed to \textbf{Docker}-containerized \textbf{Jenkins
              Pipeline}, ensuring appropriate removal of inactive accounts from Bloomberg's
          open-source GitHub repositories
    \item Developed a GitHub crawler using \textbf{Python} to scan all projects
          contributed to by Bloomberg employees over 10 years, automating contribution
          cataloging and open-source license compliance verification, increasing audited
          projects by \textbf{3x}
\end{itemize}

\section{Projects}
\subheading{Historical Landmark Image Classifier}{}{October -- November 2023}
{Python, PyTorch, Pandas, NumPy, Matplotlib, Computer Vision}{}
\begin{itemize}
    \item Designed and implemented \textbf{convolutional neural networks} for multiclass
          image classification of historical landmarks
    \item Researched \textbf{model architecture} and \textbf{data augmentation},
          employing subsampling and noise generation to improve accuracy and mitigate
          overfitting while training model with 5 convolutional layers and
          \textbf{+2,000,000} learnable parameters
    \item Utilized \textbf{transfer learning}, leveraging multiclass model to initialize
          weights for training on binary classification target problem, reducing training
          time, preventing overfitting, and fine-tuning fully-connected layers to improve
          performance
\end{itemize}
\subheading{Movie Review Prediction System}{}{September -- October 2023}
{Python, Scikit-learn, Pandas, NumPy, Gensim, Matplotlib}{}
\begin{itemize}
    \item Trained \textbf{support vector machines} capable of classifying positive and
          negative movie reviews achieving \textbf{92\%} accuracy through
          \textbf{sentiment analysis} techniques, including learned \textbf{word
              association} and \textbf{negation handling}
    \item Leveraged \textbf{word embedding association test} to determine association of
          gendered language with positive/negative adjectives in reviews, indentifying
          gender bias within dataset and resulting learned support vector machines
\end{itemize}
\subheadingLine{MST/TSP Solution Generator}{C++}{April 2023}
\begin{itemize}
    \item Developed an implementation of \textbf{Prim's algorithm} to efficiently find
          \textbf{minimum spanning trees} for complete graphs
    \item Utilized \textbf{arbitrary insertion} heuristic approach to generate
          approximate solutions for the \textbf{traveling salesperson problem} with
          quadratic time complexity, allowing for computation for \textbf{+10,000-order}
          complete graphs in seconds
    \item Created a \textbf{branch and bound} algorithm to guarantee optimal solutions to
          the traveling salesperson problem and optimized via \textbf{solution tree
              pruning}, using MST-derived upper bound, reducing runtime by \textbf{90\%}
\end{itemize}
\subheadingLine{SQL Clone}{C++}{February 2023}
\begin{itemize}
    \item Implemented a database and query command language similar to \textbf{SQL},
          including various database and table commands including table creation,
          deletion, insertion, conditional printing, conditional deletion, and inner join
    \item Incorporated \textbf{red-black trees} and \textbf{hash tables} to index tables,
          increasing efficiency of conditional print commands
    \item Utilized map indices to optimize inner join command from quadratic to linear
          time complexity
\end{itemize}
\section{Technical Skills}
\textbf{Languages}: Python, C++, Java, JavaScript/TypeScript, HTML/CSS, SQL
(SQLite), \LaTeX \\
\textbf{Tools}: Git, Docker, Jenkins, Jupyter Notebook,
MongoDB, Pandas, NumPy, Scikit-learn, Django, LDAP
\end{document}
