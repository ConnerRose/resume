\usepackage{latexsym}
\usepackage[empty]{fullpage}
\usepackage{titlesec}
\usepackage{marvosym}
\usepackage[usenames,dvipsnames]{color}
\usepackage{verbatim}
\usepackage{enumitem}
\usepackage[pdftex]{hyperref}
\usepackage{fancyhdr}
\usepackage[left=0.5in,top=0.5in,right=0.5in,bottom=0.5in]{geometry}
\usepackage{kpfonts}

\pagestyle{fancy}
\fancyhf{} % clear all header and footer fields
\fancyfoot{}
\renewcommand{\headrulewidth}{0pt}
\renewcommand{\footrulewidth}{0pt}

% Adjust margins
% \addtolength{\oddsidemargin}{-0.5in}
% \addtolength{\evensidemargin}{-0.375in}
% \addtolength{\textwidth}{1in}
% \addtolength{\topmargin}{-.5in}
% \addtolength{\textheight}{1.0in}

\urlstyle{same}

\raggedbottom
\raggedright
\setlength{\tabcolsep}{0in}
\setlength{\footskip}{5pt}

% Sections formatting
\titleformat{\section}{
    \vspace{-4pt}\scshape\raggedright\Large
}{}{0em}{}[\color{black}\titlerule \vspace{-5pt}]

%-------------------------
% Custom commands
\newcommand{\resumeItem}[2]{
    \item\small{
        \textbf{#1}{: #2 \vspace{-2pt}}
    }
}

\newcommand{\resumeSingleItem}[1]{
    \item\small{
        #1\vspace{-2pt}
    }
}

\newcommand{\resumeSubheading}[4]{
    \vspace{-1pt}\item[]
    \begin{tabular*}{\textwidth}{l@{\extracolsep{\fill}}r}
        \textbf{#1} & #2 \\
        \hspace{-10pt}\textit{\small#3} & \textit{\small #4} \\
        % \textit{\small#3} & \textit{\small #4} \\
    \end{tabular*}
    \vspace{-5pt}
}

\newcommand{\resumeProjectHeading}[1]{
    \vspace{-3pt}\item
    % {0.97\textwidth}{l@{\extracolsep{\fill}}r}
    \textbf{#1}\vspace{-5pt} }

\newcommand{\resumeSubItem}[2]{\resumeItem{#1}{#2}\vspace{-4pt}}

\renewcommand{\labelitemii}{$\circ$}

\newcommand{\resumeSubHeadingListStart}{
    \begin{itemize}[leftmargin=*]}
        \newcommand{\resumeSubHeadingListEnd}{\end{itemize}
}
\newcommand{\resumeItemListStart}{
    \begin{itemize}[leftmargin=*]
        }
        \newcommand{\resumeItemListEnd}{
    \end{itemize}
    \vspace{-5pt}}

\newcommand{\subheading}[5]{
    \ifx&#2&%
    \textbf{#1} \hfill \textbf{#3} \\ \textit{\small #4} \hfill \textit{\small #5} \\
    \else
    \textbf{#1}, \emph{#2} \hfill \textbf{#3} \\ \textit{\small #4} \hfill \textit{\small #5} \\
    \fi
}

\newcommand{\subheadingLine}[3]{
    \ifx&#2&%
    \textbf{#1} \hfill \textbf{#3} \\
    \else
    \textbf{#1}, \emph{#2} \hfill \textbf{#3} \\
    \fi
}
